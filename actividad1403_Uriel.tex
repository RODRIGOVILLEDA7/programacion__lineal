\documentclass[12pt]{article}
\usepackage[pdftex]{graphicx}
\usepackage{amsmath}
\usepackage[spanish]{babel}
\usepackage[latin1]{inputenc}
\usepackage{amssymb}
\spanishdecimal{.}

\voffset-1.0in \textheight9.75in \textwidth6.5in \hoffset-0.5in
\begin{document}

%\includegraphics[scale=0.65]{escudouaeh.pdf}\includegraphics[scale=1.0]{uni.pdf}
\pagestyle{empty}

\begin{center}


\footnotesize{\textbf{Modelos Matem�ticos Continuos}}
\\
\normalsize{Actividad 1403}
\end{center}


\textbf{Nombre: Uriel Alejandro Nolasco Hern�ndez}
\section{Actividad 1}

Partiendo del sistema del siguiente ecuaciones diferencial

\begin{align*}
    \frac{dx}{dt}&=2x+6y\\
  \frac{dy}{dt}&=-2x-5y\\
\end{align*} 
Se obtiene
\begin{align*}
    \frac{dx}{dy}&=\frac{2x+6y}{-2x-5y}
\end{align*}

Se deduce la siguiente matriz 

\begin{equation}\label{matriz}
    \mathbf{A}=\left(\begin{array}{ccc}
         2 & 6 \\
        -2 & -5 \\
    \end{array}\right)
\end{equation}

Los valores propios de la matriz son $\lambda_1={-1}$ y $\lambda_2=-2$ y sus vectores propios correspondientes respectivamente son:
\begin{equation}\label{v1}
    \mathbf{v_1}=\left(\begin{array}{ccc}
         -2\\
          1\\
 \end{array}\right)
\end{equation}
\begin{equation}\label{v2}
    \mathbf{v_2}=\left(\begin{array}{ccc}
         3\\
         -2\\
 \end{array}\right)
\end{equation}

\section{Actividad 2}

Considere el siguiente sistema de ecuaciones diferencial:

\begin{align*}
    \frac{dx}{dt}&=-3x-4y\\
  \frac{dy}{dt}&=x+y\\
\end{align*}

Se obtiene

\begin{align*}
    \frac{dx}{dy}&=\frac{-3x-4y}{x+y}
\end{align*}

La matriz asociada al sistema

\begin{equation}\label{matriz}
    \mathbf{A}=\left(\begin{array}{ccc}
         -3 & -4 \\
        1 & 1 \\
    \end{array}\right)
\end{equation}

El valor propio correspondiente a la matriz es $\lambda_1={-1}$  su vector propio correspondiente:
\begin{equation}\label{v1}
    \mathbf{v_1}=\left(\begin{array}{ccc}
         -2\\
          1\\
 \end{array}\right)
\end{equation}

\section{Actividad 3}
Los sistemas de ecuaciones son los siguientes:

\begin{align*}
    \frac{dx}{dt}&=-\frac{3x}{2}-\frac{5y}{2}\\
    \frac{dy}{dt}&=\frac{5x}{2}+\frac{3y}{2}\\
\end{align*}

Se obtiene

\begin{align*}  
  \frac{dx}{dy}&=\frac{-3x-5y}{5x+3y}
\end{align*}
La matriz asociada al sistema

\begin{equation}\label{matriz}
    \mathbf{A}=\left(\begin{array}{ccc}
         -\frac{3}{2} & -\frac{5}{2} \\
        \frac{5}{2} & \frac{3}{2} \\
    \end{array}\right)
\end{equation}

Los valores propios correspondientes a la matriz son $\lambda_1={2i}$ y $\lambda_2=-2i$ y sus vectores propios correspondientes respectivamente son:
\begin{equation}\label{v1}
    \mathbf{v_1}=\left(\begin{array}{ccc}
         \frac{-3-4i}{5}\\
          1\\
 \end{array}\right)
\end{equation}
\begin{equation}\label{v1}
    \mathbf{v_1}=\left(\begin{array}{ccc}
         \frac{-3+4i}{5}\\
         1\\
 \end{array}\right)
\end{equation}

\pagestyle{empty}
\begin{figure}[h]
    \centering
    \includegraphics[scale=1]{11.pdf}
    \label{diagrama 1}
\end{figure}
\pagestyle{empty}
\begin{figure}[h]
    \centering
    \includegraphics[scale=1]{12.pdf}
    \label{diagrama 2}
\end{figure}
\begin{figure}[h]
    \centering
    \includegraphics[scale=1]{13.pdf}
    \label{diagrama 3}
  \end{figure}

\pagestyle{empty}
\begin{figure}[h]
    \centering
    \includegraphics[scale=1]{21.pdf}
    \label{diagrama 11}
\end{figure}
\pagestyle{empty}
\begin{figure}[h]
    \centering
    \includegraphics[scale=1]{22.pdf}
    \label{diagrama 21}
\end{figure}
\begin{figure}[h]
    \centering
    \includegraphics[scale=1]{23.pdf}
    \label{diagrama 31}
  \end{figure}

\pagestyle{empty}
\begin{figure}[h]
    \centering
    \includegraphics[scale=1]{31.pdf}
    \label{diagrama 1}
\end{figure}
\pagestyle{empty}
\begin{figure}[h]
    \centering
    \includegraphics[scale=1]{32.pdf}
    \label{diagrama 2}
\end{figure}
\begin{figure}[h]
    \centering
    \includegraphics[scale=1]{33.pdf}
    \label{diagrama 3}
  \end{figure}

  

  

\end{document}
